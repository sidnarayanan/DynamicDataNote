Having computed these variables for each dataset, the popularity plot may be made. The histogram is filled for each dataset by choosing the following bin-value:
\begin{equation}\dfrac{N_\text{accesses}}{N_\text{files}\cdot \Nr}\end{equation}
The factor of $\Nf$ in the denominator is due to the fact that a single request to a dataset actually consists of a series of requests to each file in the dataset. Dividing by $N_\text{files}$ ensures that this quantity is the same for small and large datasets. The entry is given weight:
\begin{equation}
\Nr\cdot \text{size}
\end{equation}
For ease of comparing plots made under different conditions, the bin-value is normalized to the length of the time interval (in Figure 1, the unit of time is months). Currently, all datasets currently in AnalysisOps (except for \verb|USER|) are considered. All Tier 2 sites are considered, but no Tier 1 sites. Finally, the plot is normalized to have an integral of unity. The un-normalized integral can be thought of as a measure of ``average data volume'' during the interval, since it can be computed as:
\[\sum_\text{datasets} \Nr \cdot \text{size}\]
